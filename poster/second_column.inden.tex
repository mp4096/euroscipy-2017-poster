The solution is to use the so called long table format.
In this concept, each result is stored as a `line'
in a table which columns are the factors.
Now, we can use Pandas commands to create the so called \emph{pivot tables},
which basically restructure data on-demand.

Furthermore, by using Pandas, we can avoid boilerplate and verbose code.
Just use the statistics command in order to compute statistics over all required
and filtered entries.


\begin{lstlisting}
stats = df.groupby(["len_p", "fun"])\
          .describe()["runtime"]\
          .drop("count", axis=1)\
          .unstack()\
          .swaplevel(axis=1)
\end{lstlisting}

What we have seen here is the difference between the so-called \emph{long}
and \emph{short} formats.
