\textrm{\textbf{Late structuring}}
\vspace{6 pt}

The answer is yes, we \emph{can} do better.
The solution is to adopt the following two principles:
First, we should be lazy and structure our data as late as possible;
second, we should use already existing, powerful libraries such as Pandas.

\vspace{3 pt}

But how does the data look like before structuring?
Actually, quite simple:
It is a Pandas \lstinline{DataFrame}, which rows contain
factor combinations and the corresponding experiment results (see left figure).
This is the so called Pandas \emph{long format}.
So we can now rewrite our experiment code like this:

\begin{lstlisting}
import pandas as pd
df = pd.DataFrame(
    index=range(len(factors_to_investigate)),
    columns=("factor_1", "factor_2", "outcome"),
)
for i, (f_1, f_2) in enumerate(factors_to_investigate):
    df.loc[i] = (f_1, f_2, experiment(f_1, f_2))
\end{lstlisting}

\vspace{-22 pt}

And that's it! We don't need to care about the exact low-level,
internal data layout, Pandas does it all for us.
To convert our data into some structured
representation---also called \emph{wide format}---we
can use Pandas functions such as \lstinline{pivot()}, \lstinline{pivot_table()},
\lstinline{stack()}, \lstinline{unstack()}, \lstinline{groupby()}.

\vspace{3 pt}

Sure, this multitude of different reshaping functions
can be quite intimidating at the beginning.
You probably want to stick with \lstinline{groupby()} and \lstinline{describe()} at first
and learn about other functions as you go.
For example, following snippets perform structuring as it is shown in the left figure:

\begin{lstlisting}
df.groupby(["factor_1", "factor_2"]).describe()
df.groupby(["factor_1"]).describe()
df.groupby(["factor_2"]).describe()
\end{lstlisting}

\vspace{-22 pt}

And recall our example where we wanted to compute statistics
of all outcomes for which \lstinline{factor_2 == "A"}.
This is how you can do it in Pandas:

\begin{lstlisting}
df[df["factor_2"] == "A"].describe()
\end{lstlisting}

\vspace{-8 pt}

\textrm{\textbf{Final remarks}}
\vspace{6 pt}

We have seen that sometimes it is better to be lazy and
do not commit to a specific structure too soon.
This allows us to reshape and restructure data to our needs more
freely and with much less overhead.

\vspace{3 pt}

By using Pandas, we also avoid writing unnecessary boilerplate code.
Apart from statistical operations shown previously,
Pandas can import a \lstinline{DataFrame} from the most widely used formats
(CSV, Excel, JSON, \ldots) as well as export data to them.

\vspace{3 pt}

Last but not least advantage of the proposed approach is that you can make
beautiful visualisations of your data with minimal effort by using
the Seaborn package:

\begin{lstlisting}
import seaborn as sns
sns.boxplot(
    x="factor_1",
    hue="factor_2",
    y="outcome",
    data=df,
)
\end{lstlisting}
