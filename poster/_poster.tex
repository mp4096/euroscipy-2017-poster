\documentclass[12 pt, a0paper, british]{article}


% ======================================================================
% Use UTF-8 and T1 font encoding
% ======================================================================
\usepackage[utf8]{inputenc}
\usepackage[T1]{fontenc}
\usepackage{babel}

\usepackage[defaultsans]{lato}
\usepackage[rm]{merriweather}
\usepackage{FiraMono}
% ======================================================================


% ======================================================================
% Important stuff
% ======================================================================
\usepackage{geometry}
\usepackage{lipsum}
\usepackage{amssymb}
\usepackage{bm}
\usepackage{listings}
\usepackage{fontawesome}


\lstset{
  language          = Python,
  basicstyle        = \scriptsize\ttfamily,
  keywordstyle      = \color{polar night 1}\bfseries,
  commentstyle      = \color{frost 2},
  stringstyle       = \color{aurora 1},
  numbers           = left,
  numberstyle       = \tiny\color{frost 2},
  stepnumber        = 1,
  numbersep         = 5pt,
  backgroundcolor   = \color{snow storm 2},
  showspaces        = false,
  showstringspaces  = false,
  showtabs          = false,
  frame             = tb,
  rulecolor         = \color{polar night 1},
  tabsize           = 2,
  captionpos        = b,
  breaklines        = true,
  breakatwhitespace = false,
  title             = \lstname,
  escapeinside      = {@*}{*@},
  morekeywords      = {*,...}
}
% ======================================================================


% ======================================================================
% Include TikZ and TikZ libraries
% ======================================================================
\usepackage{pgfplots}
\pgfplotsset{compat=newest}
\usepackage{tikz}
\usetikzlibrary{calc}
\usetikzlibrary{arrows}
\usetikzlibrary{arrows.meta}
\usetikzlibrary{decorations}
\usetikzlibrary{decorations.pathmorphing}
% ======================================================================


% ======================================================================
% Define colors
% ======================================================================
\definecolor{polar night 0}{HTML}{2E3440}
\definecolor{polar night 1}{HTML}{3B4252}
\definecolor{polar night 2}{HTML}{434C5E}
\definecolor{polar night 3}{HTML}{4C566A}

\definecolor{snow storm 0}{HTML}{D8DEE9}
\definecolor{snow storm 1}{HTML}{E5E9F0}
\definecolor{snow storm 2}{HTML}{ECEFF4}

\definecolor{frost 0}{HTML}{8FBCBB}
\definecolor{frost 1}{HTML}{88C0D0}
\definecolor{frost 2}{HTML}{81A1C1}
\definecolor{frost 3}{HTML}{5E81AC}

\definecolor{aurora 0}{HTML}{BF616A}
\definecolor{aurora 1}{HTML}{D08770}
\definecolor{aurora 2}{HTML}{EBCB8B}
\definecolor{aurora 3}{HTML}{A3BE8C}
\definecolor{aurora 4}{HTML}{B48EAD}
% ======================================================================




\begin{document}
\pagestyle{empty}
\begin{tikzpicture}[
    remember picture,
    overlay,
    shift={(current page.north west)},
    polar night 1,
    y = -1 cm,
  ]
  \draw[snow storm 2!50, fill] (current page.north west) rectangle (current page.south east);

  % Reference points
  % \draw[aurora 0, fill] ( 0.0,   0.0) circle (1);
  % \draw[aurora 2, fill] (84.1,   0.0) circle (1);
  % \draw[aurora 3, fill] ( 0.0, 118.9) circle (1);
  % \draw[aurora 4, fill] (84.1, 118.9) circle (1);



  % Design grid -- comment out when finished
  \draw[black!40] (46.25, 0) -- +(0, 118.9) node[midway, left, black!80] {\Huge{}46.25};
  \draw[black!40] (0, 39) -- +(84.1, 0) node[midway, above, black!80] {\Huge{}39};
  \draw[black!40] (0, 74) -- +(84.1, 0) node[midway, above, black!80] {\Huge{}74};



  % Title
  \node[scale = 4.3] at (42.05, 5.3) {%
    Are you performing experiments? Then structure your data as late as possible!%
  };

  % Frame left
  \draw[polar night 1!90, very thick] (4, 5) -- (4, 113.9);
  \draw[polar night 1!80, very thick] (3, 6) -- (3, 112.9);
  \draw[polar night 1!60, very thick] (2, 7) -- (2, 111.9);

  % Frame right
  \draw[polar night 1!90, very thick] (80.1, 5) -- (80.1, 113.9);
  \draw[polar night 1!80, very thick] (81.1, 6) -- (81.1, 112.9);
  \draw[polar night 1!60, very thick] (82.1, 7) -- (82.1, 111.9);

  \node[
      text width = 15 cm,
      align = left,
      scale = 2.5,
      anchor = north west,
    ]
    at (7, 10) {\textsf{\input{first_column_indentex}}};

  \node[
      text width = 12 cm,
      align = left,
      scale = 2.5,
      anchor = north west,
    ]
    at (47.5, 40) {\textsf{\input{second_column_indentex}}};

  \node[
      text width = 15 cm,
      align = left,
      scale = 2.5,
      anchor = north west,
    ]
    at (7, 77) {\textsf{\input{third_column_indentex}}};


  \begin{scope}[xshift = 50 cm, yshift = -22 cm]
    % Flasks
\begin{scope}[xshift = 0 cm, yshift = 7 cm, y = 1 cm]
  \draw[rounded corners = 0.3 cm, fill = aurora 3]
    (2, -0.25) -- ++(-90:0.4) -- ++(-75:3.5) -- ++(180:3.25) -- ++(75:3.5) -- ++(90:0.4);
  \draw[ultra thick, rounded corners = 0.3 cm]
    (2, 0.75) -- ++(-90:1.4) -- ++(-75:3.5) -- ++(180:3.25) -- ++(75:3.5) -- ++(90:1.4);

  \draw[rounded corners = 0.3 cm, fill = aurora 2]
    (0, -1) -- ++(-90:0.4) -- ++(-75:3.5) -- ++(180:3.25) -- ++(75:3.5) -- ++(90:0.4);
  \draw[ultra thick, rounded corners = 0.3 cm]
    (0, 0) -- ++(-90:1.4) -- ++(-75:3.5) -- ++(180:3.25) -- ++(75:3.5) -- ++(90:1.4);

  \draw[aurora 2, fill] (-0.6, 0.7) circle (0.4);
  \draw[aurora 2, fill] (-0.9, 1.4) circle (0.3);
  \draw[aurora 2, fill] (-0.4, 1.9) circle (0.35);
  \draw[aurora 3, fill] (1, 1) circle (0.3);
  \draw[aurora 3, fill] (1.5, 1.6) circle (0.4);
  \draw[aurora 3, fill] (1.2, 2.4) circle (0.35);

  \node at (0.25, -5.5) {\Large{}Experiment};
\end{scope}


\draw[ultra thick, -{Latex[scale=1.2]}] (4, -6) -- (10, -6) node[midway, sloped, above] %
  {\Large{}Early structuring};


\pgfmathsetmacro{\TwoDimCellSize}{2.5}
\pgfmathsetmacro{\TwoDimHeaderSize}{1}

\begin{scope}[xshift = 12 cm, yshift = 10 cm]
  \node at ({\TwoDimCellSize*1.5 - \TwoDimHeaderSize*0.5}, -0.8) %
    {\Large{}Structured data, e.g.\ nested dicts};

  \foreach \y in {0, 1, 2} {
    % A-B-C header
    \foreach \x/\c/\s in {0/frost 1/A, 1/frost 0/B, 2/frost 3/C} {
      \draw[\c, fill]
        ({\TwoDimCellSize*\x}, {\TwoDimCellSize*\y})
        rectangle
        +(\TwoDimCellSize, \TwoDimHeaderSize)
        node[midway, polar night 1] {\s};
    }
  }
  \foreach \y/\c/\s in {0/aurora 2/I, 1/aurora 1/II, 2/aurora 0/III} {
    \draw[\c, fill]
      ({-\TwoDimHeaderSize}, {\TwoDimCellSize*\y}) rectangle +(\TwoDimHeaderSize, \TwoDimCellSize)
      node[midway, polar night 1] {\s};
  }
  \foreach \i in {0, 1, ..., 3} {
    \draw[thick]
      ({-\TwoDimHeaderSize}, {\TwoDimCellSize*\i})
      --
      +({\TwoDimCellSize*3 + \TwoDimHeaderSize}, 0);
    \draw[thick] ({\TwoDimCellSize*\i}, 0) -- +(0, {\TwoDimCellSize*3});
  }

  \begin{scope}[
      xshift = {\TwoDimCellSize*0.5 cm},
      yshift = {-(\TwoDimCellSize + \TwoDimHeaderSize)*0.5 cm},
    ]
    % I
    \node at (0, 0) {14, 15}; % A
    \node at (\TwoDimCellSize, 0) {4, 10}; % B
    \node at ({2*\TwoDimCellSize}, 0) {6, 11}; % C

    % II
    \node at (0, \TwoDimCellSize) {3, 8}; % A
    \node at (\TwoDimCellSize, \TwoDimCellSize) {\LARGE $\boldsymbol{\varnothing}$}; % B
    \node at ({2*\TwoDimCellSize}, \TwoDimCellSize) {2}; % C

    % III
    \node at (0, {2*\TwoDimCellSize}) {7, 12}; % A
    \node at (\TwoDimCellSize, {2*\TwoDimCellSize}) {1, 13}; % B
    \node at ({2*\TwoDimCellSize}, {2*\TwoDimCellSize}) {5, 9}; % C
  \end{scope}
\end{scope}


\path[
  ultra thick,
  -{Latex[scale=1.2]},
  decoration={
    zigzag,
    segment length = 32,
    amplitude = 9,
    post = lineto,
    post length = 2pt,
    },
  ] %
  (12, -1.5) edge[decorate] +(-3, 6);
\node[align = left] at (5.5, 2) %
  {\Large{}However, now it is difficult\\\Large{}to reorder the data\ldots};


\begin{scope}[xshift = 5 cm, yshift = -5 cm]
  \foreach \y in {0, 1, 2} {
    % A-B-C header
    \foreach \x/\c/\s in {0/aurora 2/I, 1/aurora 1/II, 2/aurora 0/III} {
      \draw[\c, fill]
        ({\TwoDimCellSize*\x}, {\TwoDimCellSize*\y})
        rectangle
        +(\TwoDimCellSize, \TwoDimHeaderSize)
        node[midway, polar night 1] {\s};
    }
  }
  \foreach \y/\c/\s in {0/frost 1/A, 1/frost 0/B, 2/frost 3/C} {
    \draw[\c, fill]
      ({-\TwoDimHeaderSize}, {\TwoDimCellSize*\y}) rectangle +(\TwoDimHeaderSize, \TwoDimCellSize)
      node[midway, polar night 1] {\s};
  }
  \foreach \i in {0, 1, ..., 3} {
    \draw[thick]
      ({-\TwoDimHeaderSize}, {\TwoDimCellSize*\i})
      --
      +({\TwoDimCellSize*3 + \TwoDimHeaderSize}, 0);
    \draw[thick] ({\TwoDimCellSize*\i}, 0) -- +(0, {\TwoDimCellSize*3});
  }

  \begin{scope}[
      xshift = {\TwoDimCellSize*0.5 cm},
      yshift = {-(\TwoDimCellSize + \TwoDimHeaderSize)*0.5 cm},
    ]
    % A
    \node at (0, 0) {14, 15}; % I
    \node at (\TwoDimCellSize, 0) {3, 8}; % II
    \node at ({2*\TwoDimCellSize}, 0) {7, 12}; % III

    % B
    \node at (0, \TwoDimCellSize) {4, 10}; % I
    \node at (\TwoDimCellSize, \TwoDimCellSize) {\LARGE $\boldsymbol{\varnothing}$}; % II
    \node at ({2*\TwoDimCellSize}, \TwoDimCellSize) {1, 13}; % III

    % C
    \node at (0, {2*\TwoDimCellSize}) {6, 11}; % I
    \node at (\TwoDimCellSize, {2*\TwoDimCellSize}) {2}; % II
    \node at ({2*\TwoDimCellSize}, {2*\TwoDimCellSize}) {5, 9}; % III
  \end{scope}
\end{scope}


\path[
  ultra thick,
  -{Latex[scale=1.2]},
  decoration={
    zigzag,
    segment length = 32,
    amplitude = 9,
    post = lineto,
    post length = 2pt,
    },
  ] %
  (18.5, -1.5) edge[decorate] +(3, 6);
\node at (24, 2) {\Large{}\ldots{}or to `flatten' it};


% 1D summary, A-B-C only
\pgfmathsetmacro{\OneDimLineHeight}{2.5}
\pgfmathsetmacro{\OneDimLineWidth}{5.5}
\pgfmathsetmacro{\OneDimHeaderWidth}{1}

\begin{scope}[xshift = 20 cm, yshift = -5 cm]
  \foreach \y/\c/\s in {0/frost 1/A, 1/frost 0/B, 2/frost 3/C} {
    \draw[\c, fill]
      ({-\OneDimHeaderWidth}, {\OneDimLineHeight*\y}) rectangle +(\OneDimHeaderWidth, \OneDimLineHeight)
      node[midway, polar night 1] {\s};
  }
  \foreach \i in {0, 1, ..., 3} {
    \draw[thick] ({-\OneDimHeaderWidth}, {\OneDimLineHeight*\i}) -- +({\OneDimLineWidth + \OneDimHeaderWidth}, 0);
  }
  \draw[thick] (0, 0) -- +(0, {3*\OneDimLineHeight});

  \begin{scope}[xshift = 0.5 cm, yshift = {-\OneDimLineHeight*0.5 cm}, right]
    \node at (0, 0) {3, 7, 8, 12, 14, 15}; % A
    \node at (0, \OneDimLineHeight) {1, 4, 10, 13}; % B
    \node at (0, {2*\OneDimLineHeight}) {2, 5, 6, 9, 11}; % C
  \end{scope}
\end{scope}

  \end{scope}

  \begin{scope}[xshift = 12 cm, yshift = -52.5 cm]
    \Large

% Flasks
\begin{scope}[xshift = 0 cm, yshift = 7 cm, y = 1 cm]
  \draw[rounded corners = 0.3 cm, fill = aurora 3]
    (2, -0.25) -- ++(-90:0.4) -- ++(-75:3.5) -- ++(180:3.25) -- ++(75:3.5) -- ++(90:0.4);
  \draw[ultra thick, rounded corners = 0.3 cm]
    (2, 0.75) -- ++(-90:1.4) -- ++(-75:3.5) -- ++(180:3.25) -- ++(75:3.5) -- ++(90:1.4);

  \draw[rounded corners = 0.3 cm, fill = aurora 2]
    (0, -1) -- ++(-90:0.4) -- ++(-75:3.5) -- ++(180:3.25) -- ++(75:3.5) -- ++(90:0.4);
  \draw[ultra thick, rounded corners = 0.3 cm]
    (0, 0) -- ++(-90:1.4) -- ++(-75:3.5) -- ++(180:3.25) -- ++(75:3.5) -- ++(90:1.4);

  \draw[aurora 2, fill] (-0.6, 0.7) circle (0.4);
  \draw[aurora 2, fill] (-0.9, 1.4) circle (0.3);
  \draw[aurora 2, fill] (-0.4, 1.9) circle (0.35);
  \draw[aurora 3, fill] (1, 1) circle (0.3);
  \draw[aurora 3, fill] (1.5, 1.6) circle (0.4);
  \draw[aurora 3, fill] (1.2, 2.4) circle (0.35);

  \node at (0.25, -5.5) {Experiment};
\end{scope}


\draw[ultra thick, -{Latex[scale=1.2]}] (3.5, -6) -- (10.5, -6) node[midway, sloped, above] %
  {No structuring};


\begin{scope}[xshift = 12 cm, yshift = 12 cm]
  \pgfmathsetmacro{\LongTableCellHeight}{1}

  % Colourful cells
  \foreach \y/\c/\s in {%
    0/frost 0/B,%
    1/frost 3/C,%
    2/frost 1/A,%
    3/frost 0/B,%
    4/frost 3/C,%
    5/frost 3/C,%
    6/frost 1/A,%
    7/frost 1/A,%
    8/frost 3/C,%
    9/frost 0/B,%
    10/frost 3/C,%
    11/frost 1/A,%
    12/frost 0/B,%
    13/frost 1/A,%
    14/frost 1/A%
    } {
    \draw[\c, fill]
      (4.5, {\LongTableCellHeight*\y}) rectangle +(3, \LongTableCellHeight)
      node[midway, polar night 1] {\s};
  }
  \foreach \y/\c/\s in {%
    0/aurora 0/III,%
    1/aurora 1/II,%
    2/aurora 1/II,%
    3/aurora 2/I,%
    4/aurora 0/III,%
    5/aurora 2/I,%
    6/aurora 0/III,%
    7/aurora 1/II,%
    8/aurora 0/III,%
    9/aurora 2/I,%
    10/aurora 2/I,%
    11/aurora 0/III,%
    12/aurora 0/III,%
    13/aurora 2/I,%
    14/aurora 2/I%
    } {
    \draw[\c, fill]
      (1.5, {\LongTableCellHeight*\y}) rectangle +(3, \LongTableCellHeight)
      node[midway, polar night 1] {\s};
  }

  % Horizontal lines
  \foreach \y in {0, 1, ..., 15} {
    \draw[thick] (0, {\LongTableCellHeight*\y}) -- +(9.5, 0);
  }
  % Dots
  \foreach \y in {1, 2, ..., 15} {
    \node at (0.5, {\LongTableCellHeight*(\y - 0.5)}) {\y};
    \node at (8.5, {\LongTableCellHeight*(\y - 0.5)}) {\ldots};
  }
\end{scope}


\draw[ultra thick, -{Latex[scale=1.2]}] (13.5, 4) -- +(-10, 6) node[midway, sloped, above] %
  {bla};
\draw[ultra thick, -{Latex[scale=1.2]}] (17.0, 4) -- +(-2, 7.2) node[midway, sloped, above] %
  {No structuring};
\draw[ultra thick, -{Latex[scale=1.2]}] (20.5, 4) -- +(8, 7.2) node[midway, sloped, above] %
  {No structuring};


% 2D summary
\pgfmathsetmacro{\TwoDimCellSize}{3}
\pgfmathsetmacro{\TwoDimHeaderSize}{1}

\begin{scope}[xshift = -2 cm, yshift = -11 cm]
  \foreach \y in {0, 1, 2} {
    % A-B-C header
    \foreach \x/\c/\s in {0/frost 1/A, 1/frost 0/B, 2/frost 3/C} {
      \draw[\c, fill]
        ({\TwoDimCellSize*\x}, {\TwoDimCellSize*\y})
        rectangle
        +(\TwoDimCellSize, \TwoDimHeaderSize)
        node[midway, polar night 1] {\s};
    }
  }
  \foreach \y/\c/\s in {0/aurora 2/I, 1/aurora 1/II, 2/aurora 0/III} {
    \draw[\c, fill]
      ({-\TwoDimHeaderSize}, {\TwoDimCellSize*\y}) rectangle +(\TwoDimHeaderSize, \TwoDimCellSize)
      node[midway, polar night 1] {\s};
  }
  \foreach \i in {0, 1, ..., 3} {
    \draw[thick]
      ({-\TwoDimHeaderSize}, {\TwoDimCellSize*\i})
      --
      +({\TwoDimCellSize*3 + \TwoDimHeaderSize}, 0);
    \draw[thick] ({\TwoDimCellSize*\i}, 0) -- +(0, {\TwoDimCellSize*3});
  }

  \begin{scope}[
      xshift = {\TwoDimCellSize*0.5 cm},
      yshift = {-(\TwoDimCellSize + \TwoDimHeaderSize)*0.5 cm},
    ]
    % I
    \node at (0, 0) {14, 15}; % A
    \node at (\TwoDimCellSize, 0) {4, 10}; % B
    \node at ({2*\TwoDimCellSize}, 0) {6, 11}; % C

    % II
    \node at (0, \TwoDimCellSize) {3, 8}; % A
    \node at (\TwoDimCellSize, \TwoDimCellSize) {\LARGE $\boldsymbol{\varnothing}$}; % B
    \node at ({2*\TwoDimCellSize}, \TwoDimCellSize) {2}; % C

    % III
    \node at (0, {2*\TwoDimCellSize}) {7, 12}; % A
    \node at (\TwoDimCellSize, {2*\TwoDimCellSize}) {1, 13}; % B
    \node at ({2*\TwoDimCellSize}, {2*\TwoDimCellSize}) {5, 9}; % C
  \end{scope}
\end{scope}


% 1D summary, I-II-II only
\pgfmathsetmacro{\OneDimLineHeight}{2.5}
\pgfmathsetmacro{\OneDimLineWidth}{6.5}
\pgfmathsetmacro{\OneDimHeaderWidth}{1}

\begin{scope}[xshift = 13 cm, yshift = -12 cm]
  \foreach \y/\c/\s in {0/aurora 2/I, 1/aurora 1/II, 2/aurora 0/III} {
    \draw[\c, fill]
      ({-\OneDimHeaderWidth}, {\OneDimLineHeight*\y}) rectangle +(\OneDimHeaderWidth, \OneDimLineHeight)
      node[midway, polar night 1] {\s};
  }
  \foreach \i in {0, 1, ..., 3} {
    \draw[thick] ({-\OneDimHeaderWidth}, {\OneDimLineHeight*\i}) -- +({\OneDimLineWidth + \OneDimHeaderWidth}, 0);
  }
  \draw[thick] (0, 0) -- +(0, {3*\OneDimLineHeight});

  \begin{scope}[xshift = 0.5 cm, yshift = {-\OneDimLineHeight*0.5 cm}, right]
    \node at (0, 0) {4, 6, 10, 11, 14, 15}; % I
    \node at (0, \OneDimLineHeight) {2, 3, 8}; % II
    \node at (0, {2*\OneDimLineHeight}) {1, 5, 9, 7, 12, 13}; % III
  \end{scope}
\end{scope}


% 1D summary, A-B-C only
\begin{scope}[xshift = 26 cm, yshift = -12 cm]
  \foreach \y/\c/\s in {0/frost 1/A, 1/frost 0/B, 2/frost 3/C} {
    \draw[\c, fill]
      ({-\OneDimHeaderWidth}, {\OneDimLineHeight*\y}) rectangle +(\OneDimHeaderWidth, \OneDimLineHeight)
      node[midway, polar night 1] {\s};
  }
  \foreach \i in {0, 1, ..., 3} {
    \draw[thick] ({-\OneDimHeaderWidth}, {\OneDimLineHeight*\i}) -- +({\OneDimLineWidth + \OneDimHeaderWidth}, 0);
  }
  \draw[thick] (0, 0) -- +(0, {3*\OneDimLineHeight});

  \begin{scope}[xshift = 0.5 cm, yshift = {-\OneDimLineHeight*0.5 cm}, right]
    \node at (0, 0) {3, 7, 8, 12, 14, 15}; % A
    \node at (0, \OneDimLineHeight) {1, 4, 10, 13}; % B
    \node at (0, {2*\OneDimLineHeight}) {2, 5, 6, 9, 11}; % C
  \end{scope}
\end{scope}

  \end{scope}


  % Footer -- author info, license
  \begin{scope}[xshift = 42.05 cm, yshift = -113.9 cm]
    \node[
      scale = 1.3,
      rounded corners = 0.4 cm,
      fill = snow storm 2,
      draw,
      rectangle,
      inner sep = 0.5 cm,
      thick,
      ] at (0, -1) %
      {Mikhail Pak, 2017 --- \faGithub\ \texttt{mp4096} --- \faCreativeCommons\ \texttt{BY-NC-SA 4.0}};
  \end{scope}
\end{tikzpicture}
\end{document}
