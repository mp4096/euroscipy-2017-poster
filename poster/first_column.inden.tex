Doing experiments

Experiments, e.g.\ benchmarking algorithms or investigating how some factors influence some outcomes.

Terms: Factors, outcomes

Most obvious thing is to make use of Python dicts and build up a
hierarchy

\begin{lstlisting}
def experiment(factor_a, factor_b):
    # ... do awesome stuff
    return result

# Initialise `data` somehow
for f_a, f_b in factors_collection:
    data[f_a][f_b] = experiment(f_a, f_b)
\end{lstlisting}

However, this leads to some problems

The most obvious one is that this dict of dicts (of dicts etc) is not
flexible: one cannot easily reorder data.
For instance, if we want to index first over the second factor and then over the first,
this would require a tedious double for loop.

Furthermore, operating on these data structures requires a lot of boilerplate:
suppose we want to compute statistics over all results for factor 1 == A.
This will require a lot of boilerplate.

To sum this up, this solution is cumbersome and verbose. Can we do better?
