\textrm{\textbf{What is an experiment?}}
\vspace{6 pt}

In science, we do a lot of \emph{experiments},
e.g.\ quantifying chemical or physical phe\-nomena,
benchmarking algorithms or performing social surveys.
Often we would manipulate some \emph{factors}
in order to see how they influence the experiment \emph{outcome}.

\vspace{3 pt}

For instance, imagine we perform an experiment with two factors,
\lstinline{factor_1} and \lstinline{factor_2}.
They can take values \lstinline{"I"}, \lstinline{"II"}, \lstinline{"III"}
or \lstinline{"A"}, \lstinline{"B"}, \lstinline{"C"} respectively.
We can abstract this experiment as a Python function:

\begin{lstlisting}
def perform_experiment(factor_1, factor_2):
    # ...
    return outcome
\end{lstlisting}

\vspace{-8 pt}

\textrm{\textbf{Early structuring}}
\vspace{6 pt}

Now it's fun time! We actually perform this experiment
for factor combinations specified in \lstinline{factors_to_investigate}.
But how do we store the results?
A naïve approach would be to make use of Python dicts
and build up a nested dict structure.
Often we want to test a particular factor combination multiple times,
e.g.\ when we want to investigate how consistent the outcomes are.
Hence, we initialise our dict of dicts with an empty list
and append the experiment results if any.

\begin{lstlisting}
data = {
  f_1: {f_2: [] for f_2 in ["A", "B", "C"]}
  for f_1 in ["I", "II", "III"]
}
for f_1, f_2 in factors_to_investigate:
    data[f_1][f_2].append(experiment(f_1, f_2))
\end{lstlisting}

\vspace{-22 pt}

Although ostensibly simple, this approach leads to significant problems.
The most obvious one is that the nested dicts are not flexible.
Reordering or reshaping operations,
such as making \lstinline{factor_2} the first index or flattening data (see the right figure),
would have to be written \emph{ad hoc} and they would be very verbose.
This inflexibility of the data format can pose an insurmountable overhead during
rapid, iterative development of scientific code.

\vspace{3 pt}

Coincidentally, early structuring makes processing older data significantly more difficult.
Imagine you want to manipulate an additional factor \lstinline{factor_3}.
You rewrite your postprocessing and visualisation code to accomodate for the additional nesting level,
but now you have to either modify all previously stored results to use them with the new code
or maintain multiple codebases, one for the old structure and one for the new one.

\vspace{3 pt}

Finally, any mathematical operations on such data structures require
a lot of brittle, verbose code.
Suppose we want to compute statistics (mean, standard deviation etc.)
over all results for \lstinline{factor_2 == "A"}:

\begin{lstlisting}
import numpy as np
flattened = []
for _, subdict in data.items():
    for k, v in subdict.items():
        if k == "A":
            flattened.extend(v)
flattened = np.array(flattened)
print(flattened.mean(), flattened.std())
\end{lstlisting}

\vspace{-22 pt}

Just imagine how much you will have to refactor here.
Can we do better?
